% ****** Start of file apssamp.tex ******
%
%   This file is part of the APS files in the REVTeX 4.1 distribution.
%   Version 4.1r of REVTeX, August 2010
%
%   Copyright (c) 2009, 2010 The American Physical Society.
%
%   See the REVTeX 4 README file for restrictions and more information.
%
% TeX'ing this file requires that you have AMS-LaTeX 2.0 installed
% as well as the rest of the prerequisites for REVTeX 4.1
%
% See the REVTeX 4 README file
% It also requires running BibTeX. The commands are as follows:
%
%  1)  latex apssamp.tex
%  2)  bibtex apssamp
%  3)  latex apssamp.tex
%  4)  latex apssamp.tex
%
\documentclass[
reprint,
superscriptaddress,
%groupedaddress,
%unsortedaddress,
%runinaddress,
%frontmatterverbose, 
%preprint,
showpacs,
%preprintnumbers,
%nofootinbib,
%nobibnotes,
%bibnotes,
amsmath,
amssymb,
aps,
pra,
%prb,
%rmp,
%prstab,
%prstper,
%floatfix,
longbibliography
]{revtex4-1}

\usepackage{graphicx}% Include figure files
\usepackage{dcolumn}% Align table columns on decimal point
\usepackage{bm}% bold math
\usepackage{color}% Allows color text
%\usepackage{hyperref}% add hypertext capabilities
%\usepackage[mathlines]{lineno}% Enable numbering of text and display math
%\linenumbers\relax % Commence numbering lines
\usepackage{ulem}% allows strike-out using sout
\usepackage[caption=false]{subfig}

%\usepackage[showframe,%Uncomment any one of the following lines to test 
%%scale=0.7, marginratio={1:1, 2:3}, ignoreall,% default settings
%%text={7in,10in},centering,
%%margin=1.5in,
%%total={6.5in,8.75in}, top=1.2in, left=0.9in, includefoot,
%%height=10in,a5paper,hmargin={3cm,0.8in},
%]{geometry}

\providecommand{\auedit}[1]{#1}
\newif\ifcmnt
%  Use \cmntfalse to not see comments when it is latex'ed
%\cmntfalse
%  Use \cmnttrue to see the comments
\cmnttrue

\ifcmnt
    \providecommand{\aucmnt}[1]{#1}
\else
    \providecommand{\aucmnt}[1]{}
\fi
\newcommand{\HV}[1]{\auedit{\textcolor{blue}{#1}}}
\newcommand{\HVc}[1]{\aucmnt{\textcolor{blue}{[HV: #1]}}}
\newcommand{\HVs}[1]{\auedit{\textcolor{blue}{\sout{#1}}}}
\newcommand{\SG}[1]{\auedit{\textcolor{magenta}{#1}}}
\newcommand{\SGc}[1]{\aucmnt{\textcolor{magenta}{[SG: #1]}}}
\newcommand{\SGs}[1]{\auedit{\textcolor{magenta}{\sout{#1}}}}
\newcommand{\LS}[1]{\auedit{\textcolor{green}{#1}}}
\newcommand{\LSc}[1]{\aucmnt{\textcolor{green}{[GS: #1]}}}
\newcommand{\TC}[1]{\auedit{\textcolor{red}{#1}}}

\newcommand{\rhotrue}{\rho_{\text{true}}}

\begin{document}

% \preprint{APS/123-QED}

\title{Quadrature Histograms in Maximum Likelihood Quantum State Tomography}% Force line breaks with \\
%\thanks{A footnote to the article title}%
\author{J. L. E. Silva}
\affiliation{Departamento de Engenharia de Teleinform\'atica, Universidade Federal do Cear\'a, Fortaleza, Cear\'a, 60440, Brazil}
\author{H. M. Vasconcelos}
\email{hilma@ufc.br}
\affiliation{Departamento de Engenharia de Teleinform\'atica, Universidade Federal do Cear\'a, Fortaleza, Cear\'a, 60440, Brazil}
\affiliation{Applied and Computational Mathematics Division, National Institute of Standards and Technology, Boulder, Colorado, 80305, USA}
\author{S. Glancy}
\affiliation{Applied and Computational Mathematics Division, National Institute of Standards and Technology, Boulder, Colorado, 80305, USA}

%\collaboration{MUSO Collaboration}%\noaffiliation

\date{\today}% It is always \today, today,
             %  but any date may be explicitly specified

\begin{abstract}
Quantum state tomography (QST) aims to determine the quantum state of a system from measured data and is an essential tool for quantum information. 
When dealing with quantum states of light, QST is done by measuring  quantum  noise  statistics  of the  field  amplitudes  at different  optical  
phases using homodyne detection. The quadrature-phase homodyne measurement outputs a continuous variable, but we can histogram the continuous 
measurements and make the statistical estimation faster without losing too much information. This paper investigate different ways to determine the 
quadrature histograms for optical homodyne QST.

\end{abstract}

\pacs{
03.65.Wj, %State reconstruction, quantum tomography
03.67.-a, % Quantum information
42.50.Dv %Quantum state engineering and measurements in quantum optics
} % PACS, the Physics and Astronomy Classification Scheme.
%\keywords{Suggested keywords}%Use showkeys class option if keyword
                              %display desired
\maketitle

%\tableofcontents

\section{Introduction}
\label{intro}
Quantum information science and engineering is now at the point where 
rudimentary quantum computers are available in the laboratory and 
commercially~\cite{kandala2017,Linke2017,Monk2017,Denchev2016}.
Consequently, precise reconstruction and diagnostic tools used to estimate 
quantum states~\cite{Vogel1989, Smithey1993, Dunn1995, Banaszek1999, Banaszek2000, White2002, Ourjoumtsev2007, Neergaard2006}, 
processes~\cite{Chuang1997, Poyatos1997, Altepeter2003, Dariano1998, Nielsen1998, Mitchell2003, Obrien2004,Kupchak2015}, and 
measurements~\cite{Luis1999, Fiurasek2001, Dariano2004, Lundeen2009} are fundamental.

Quantum tomography techniques for quantum states of light became a subject of
major interest in recent years, since quantum light sources are essential
for implementations of continuous-variable (CV) quantum computation~\cite{Lloyd1999, Gottesman2001, Bartlett2002, Jeong2002, Ralph2003}. 
These source are also extensively exploited in quantum criptography~\cite{Ralph1999, Hillery2000, Silberhorn2002, Pirandola2008, Luiz2017}, 
quantum metrology~\cite{Eberle2010, Demkowicz2013}, state teleportation~\cite{Vaidman1994, Braunstein1998, He2015}, dense 
coding~\cite{Braunstein2000, Lee2014} and cloning~\cite{Cerf2000, Braunstein2001}. 

In quantum state tomography, we perform a large number of experimental measurements on
a collection of quantum systems, all prepared in a same unknown state. The goal is to
estimate this unknown state from the experimental measurements results. This estimation
can be done from the experimental statistical data by different methods. In here
we will be dealing with Maximum Likelihood estimation, that finds among all possible
states, the one which maximizes the probability of obtaining the experimental data set
in hand. 

Quantum homodyne tomography is one of the most popular optical tomography techniques
available. It rapidly became a versatile tool and has been applied in many different quantum optics experimental settings since 
it was proposed by Vogel and Risken in 1989~\cite{Vogel1989} and first implemented by Smithey \textit{et al.} in 1993~\cite{Smithey1993}. 
This technique permits to characterize a light quantum state by means of the electric field quantum noise statistics 
collected through multiple phase-sensitive measurements. 

A homodyne measurement generates a continuous value. While no data binning is necessarily
needed, we believe that the loss due to binning may be insignificant. On the other hand,
discretization of the data by binning it reduces considerably the number of data, expediting the reconstruction 
algorithm. But how can we estimate the quadrature bin width, such that the bins are not too small nor too big? 
Bins should not be too small in order to guarantee a save in calculation time and memory. 
And bins should not be too large in order to avoid a lack of resolution, making the histogram a poor representation 
of the underlying distribution shape.
 
In this paper, we use numerical experiments to simulate optical homodyne tomography of quantum optical states and 
perform maximum likelihood tomography on the data with and without binning. When choosing a quadrature bin width, 
we use and compare two different ways: Scott's rule~\cite{Scott2010} and equation (5.136) from Leonhardt's 
book~\cite{Leonhardt1997}. The paper is divide as follow: in Section \ref{MLE} we review maximum 
likelihood in homodyne tomography.  In Section \ref{numerical-experiments} we describe our numerical experiments and present 
our results. In Section \ref{conclusion} we discuss the interpretation of our results and make some concluding remarks.


\section{Maximum likelihood in homodyne tomography}
\label{MLE}
Let us consider $N$ quantum systems, each of them prepared in an optical state described by a density matrix $\rhotrue$. In each 
experimental run, we measure the field quadrature at different phases $\theta$ of a local oscillator, i.e. a reference system prepared in 
a high amplitude coherent state. Each measurement is associated with an observable $\hat{X}_{\theta} = \hat{X} \cos \theta + \hat{P} \sin \theta$, 
where $\hat{X}$ and $\hat{P}$ are the position and momentum operators, respectively. For a given phase $\theta$, we measure a quadrature value $x$, 
resulting on a data set $\{(\theta_i, x_i)\}$.

The outcome of the $i$-th measurement is described by a positive-operator-valued measure (POVM) element $\Pi (x_i|\theta_i) = \Pi_i$. 
Given the data set \{$(\theta_{i}, x_i): i = 1, ..., N$\}, we can write the likelihood of a candidate density matrix $\rho$ as
\begin{eqnarray}
\mathcal{L} (\rho)= \prod_{i=1}^{N} \mathrm{Tr} (\Pi_i \rho),
\label{eq-likelihood}
\end{eqnarray}
where $\mathrm{Tr}(\rho \Pi_i)$ is the probability, when measuring with phase $\theta_i$, to obtain outcome $x_i$, 
according to the candidate density matrix $\rho$.

MLE searches within the density matrix space the one that maximizes the likelihood in~(\ref{eq-likelihood}). Equivalently, 
it usually is more convenient to maximize the logarithm of the likelihood (the ``log-likelihood''):
\begin{eqnarray}
L (\rho) = \ln \mathcal{L} (\rho)= \sum_{i=1}^{N} \ln [\mathrm{Tr} (\Pi_i \rho)],
\end{eqnarray} 
which is maximized by the same density matrix as the likelihood. The MLE is essentially a function optimization problem, and 
since the log-likelihood function is concave, the convergence to an unique solution will be achieved by most iterative optimization methods.

In our numerical simulations, we use an algorithm for likelihood maximization that begins with interactions of the $R\rho R$ algorithm~\cite{Rehacek2007} 
followed by iterations of a regularized gradient ascent algorithm (RGA). The main reason to switch from one algorithm to another is the fact that an 
expressive slow-down is observed in the $R\rho R$ algorithm after about $(n+1)^2/4$ iterations. In the RGA, $\rho^{(k+1)}$ is parametrized as 
\begin{equation}
  \rho^{(k+1)}=\frac{\left(\sqrt{\rho^{(k)}}+A\right)\left(\sqrt{\rho^{(k)}}+A^{\dagger}\right)}{\mathrm{Tr}\left[\left(\sqrt{\rho^{(k)}}+A\right)\left(\sqrt{\rho^{(k)}}+A^{\dagger}\right)\right]},
\label{eq-rho-k+1}
\end{equation}
where $\rho^{(k)}$ is the density found by the last interaction of $R \rho R$, and $A$ may be any complex matrix of the same dimensions as $\rho$. 
Eq.~(\ref{eq-rho-k+1}) ensures that $\rho^{(k+1)}$ is a physical density matrix for any choosed $A$. The matrix $A$ should maximize the quadratic
approximation of the log-likelihood subject to $\text{Tr}(AA^{\dagger})\leq u$, where $u$ is a positive number adjusted by the algorithm to guarantee 
that the log-likelihood increases with each iteration. To halt the interactions, we use the stopping criterion of \cite{Glancy2012}, 
$L(\rho_{\text{ML}})-L(\rho^{(k)})\leq 0.2$, where $L(\rho_{\text{ML}})$ is the maximum of the log-likelihood.


\section{Numerical experiments}
\label{numerical-experiments}
Our numerical experiments simulate single mode optical homodyne
measurements~\cite{Lvovsky2009} of Gaussian cat and squeezed vacuum states. 
Each considered state is represented by a density matrix $\rho_{\mathrm{true}}$ in an $n$ photon basis. 

In order to calculate the probability to obtain homodyne measurement outcome $x$, when measuring state $\rho_{\mathrm{true}}$ 
with phase $\theta$, we need to represent $\Pi (x|\theta)$ in the $n$ photon basis considered. If $|x\rangle$ is the photon 
number basis representation of the x-quadrature eigenstate with eigenvalue $x$, and $U(\theta)$ is the phase evolution unitary 
operator, then for an ideal homodyne measurement, we have $\Pi (x|\theta) = U(\theta)^{\dagger}|x\rangle \langle x| U(\theta)$. 
Moreover, it is more realistic to consider that homodyne detectors suffer from photon loss, by including that loss in the POVM elements. 
In this case, the projector operator is replaced by $\Pi (x|\theta) = \sum_{i=1}^{n} E_i(\eta)^{\dagger} U(\theta)^{\dagger}|x\rangle \langle x| U(\theta) E_i(\eta)$, 
where $\eta = 0.9$, which is typical for state-of-the-art homodyne detectors, is used in all
simulations. Using this strategy, we are able to estimate the state of the system before the occurrence of the regarded loss. We use rejection
sampling from the distribution given by $P(x|\theta)$ to guarantee random samples of homodyne measurement results~\cite{Kennedy1980}.

To choose the phases at which the homodyne measurements are performed, we divide the upper-half-circle evenly among $m$ phases 
between 0 and $\pi$ and measure $N/m$ times at each phase, where $N$ is the total number of measurements. In all simulations, we 
use $m=20$ and $N = 20,000$. To secure a single maximum of the likelihood function, we need an informationally complete set of measurement 
operators, which can be obtained if we use $n+1$ different phases to reconstruct a state that contains at most $n$ photons~\cite{Leonhardt1997}. 

To quantify how similar a reconstructed state, $\rho$, is of a true state, $\rho_{\mathrm{true}}$, we use the fidelity, defined by:
\begin{eqnarray}
F = Tr \sqrt{\rho^{1/2}\, \rho_{\mathrm{true}} \, \rho^{1/2}}.
\end{eqnarray}  
The fidelity shown in the graphs are obtained by calculating the arithmetic mean of the 100 fidelities, since we reconstruct each state 100 times, 
each time obtaining the fidelity between the reconstructed state and the true state. The uncertainty in each fidelity estimate 
(shown as error bars in the figures) is the standard deviation of the mean of the fidelity.

We calculate and compare the fidelity between the reconstructed state and the true state for three different situations: (i) the state is 
reconstructed using the continuous values of homodyne measurement results, that is without binning; (ii) the state is reconstructed 
using homodyne measurement data binning with a bin width given by Scott's rule~\cite{Scott2010}; and (iii) the state is reconstructed 
using homodyne measurement data binning with a bin width suggested by Leonhardt in~\cite{Leonhardt1997}.

In 1979 Scott derived a formula for the asymptotically optimal bin width:
\begin{eqnarray}
h^{\star} = \left[ \frac{6}{s \int_{-\infty}^{\infty} f'(x)^2 dx} \right]^{1/3},
\label{eq-hstar}
\end{eqnarray}
where $f(x)$ is a continuous probability density function with two
continuous bounded derivatives and $s$ is the sample size. For a Gaussian probability density, we have
\begin{eqnarray}
\int_{-\infty}^{\infty} f'(x)^2 dx = \frac{1}{4 \sqrt{\pi} \sigma ^3},
\label{eq-intnormaldist}
\end{eqnarray}
where $\sigma$ is the standard deviation. Combining Eq. (\ref{eq-hstar}) and Eq. (\ref{eq-intnormaldist}), we obtain the 
optimal bin width for normal data distribution:
\begin{eqnarray}
h = 3.5 \, \sigma \, s^{-1/3}.
\end{eqnarray}
This formula is known as Scott's rule, and is optimal if the data is close to being normally distributed, but is also 
appropriate for most other distributions.


\section{Estimating mean photon number}
\SGc{Here are some quick and rough notes explaining how we can
  estimate the mean number of photons.  They will need a lot of
  revision before they are really part of the paper.  I am only
  putting them here, because it seemed like a convenient way to share
  with Hilma and Leonardo.}  In order to use Leonhardt's advice for
choosing the histogram bin width, we need to estimate the mean number
$\langle n \rangle$ of photons in the measured state from the
phase-quadrature data set.  To find an estimator, we first compute the
mean value of $(\hat{X}_{\theta})^{2}$, averaged over $\theta$, treating
$\theta$ as if it is random and uniformly distributed bewtween $0$ and
$\pi$.
\begin{equation}
\langle (\hat{X}_{\theta})^{2} \rangle = \langle \hat{X}^{2}\cos^{2}\theta + (\hat{X}\hat{P}+\hat{P}\hat{X})\cos\theta\sin\theta + \hat{P}^{2}\sin^{2}\theta \rangle
\end{equation}
The phase $\theta$ is independent of $\hat{X}$ and $\hat{P}$, so we can compute the expectation over $\theta$ as
\begin{align}
\langle (\hat{X}_{\theta})^{2} \rangle &= \Big\langle \int_{0}^{\pi} (\hat{X}^{2}\cos^{2}\theta + (\hat{X}\hat{P}+\hat{P}\hat{X})\cos\theta\sin\theta \nonumber \\
 & \qquad \qquad + \hat{P}^{2}\sin^{2}\theta) \mathrm{Prob}(\theta) \mathrm{d}\theta \Big\rangle \\
\langle (\hat{X}_{\theta})^{2} \rangle &= \Big\langle \int_{0}^{\pi} (\hat{X}^{2}\cos^{2}\theta + (\hat{X}\hat{P}+\hat{P}\hat{X})\cos\theta\sin\theta \nonumber \\
 & \qquad \qquad + \hat{P}^{2}\sin^{2}\theta) \frac{1}{\pi} \mathrm{d}\theta \Big\rangle \\
 &= \left\langle (\hat{X}^{2}\frac{\pi}{2} + \hat{P}^{2}\frac{\pi}{2})\frac{1}{\pi} \right\rangle \\
 &= \frac{1}{2}\left\langle \hat{X}^{2} + \hat{P}^{2} \right\rangle
\end{align}
Leonardo showed that
\begin{equation}
\hat{n} = \frac{1}{2}\left(\hat{X}^{2}+\hat{P}^{2}-1\right).
\end{equation}
Therefore
\begin{align}
\langle\hat{n}\rangle &= \frac{1}{2}\left(\langle\hat{X}^{2}+\hat{P}^{2}\rangle-1\right) \\
\langle\hat{n}\rangle &= \langle \hat{X}_{\theta}^{2}\rangle-\frac{1}{2}.
\end{align}
Thus calculating the expectation value of $\hat{X}_{\theta}^{2}$ gives us the mean number of photons.  We can estimate $\langle \hat{n} \rangle$ by computing
\begin{equation}
\overline{\langle \hat{n} \rangle} = \frac{1}{N} \sum_{i=1}^{N}x_{i}^{2} - \frac{1}{2}.
\end{equation}
Note that when $\theta$ is uniformly distributed over $[0,\pi)$, the values of $\theta$ are not needed to compute $\overline{\langle \hat{n} \rangle}$.

To compute this estimate, we need a new Matlab function that takes the
quadrature data as input and computes the square of each quadrature.
Then we simply average the squares and subtract 1/2.  Then, instead of
using the true value of $\langle \hat{n} \rangle$ as we have been
doing, we should use $\overline{\langle \hat{n} \rangle}$ in
Leonhardt's formula to choose bin size.



\section{Conclusion}
\label{conclusion}



\begin{acknowledgments}
We thank Kevin Coakley, Adam Keith, and Emanuel Knill for helpful
comments on the manuscript.  H. M. Vasconcelos thanks the Schlumberger Foundation's Faculty for 
the Future program for financial support. J. L. E. Silva thanks Coordena\c c\~ao de Aperfei\c coamento de 
Pessoal de N\'ivel Superior (CAPES) for financial support. This work includes contributions of the National 
Institute of Standards and Technology, which are not subject to U.S. copyright.
\end{acknowledgments}


% BibTeX users please use one of
%\bibliographystyle{spbasic}      % basic style, author-year citations
%\bibliographystyle{spmpsci}      % mathematics and physical sciences
%\bibliographystyle{spphys}       % APS-like style for physics
% Scott: My LaTeX does not know about spphys, but it is not necessary
% to specify a bibliography style.  revtex should authomatically use
% the correct style based on the documentclass.
\bibliography{histogram}   % name your BibTeX data base


% Non-BibTeX users please use%\begin{thebibliography}{}
%
% and use \bibitem to create references. Consult the Instructions
% for authors for reference list style.


%\end{thebibliography}


\end{document}

%
% ****** End of file apssamp.tex ******

%%% Local Variables:
%%% mode: latex
%%% TeX-master: t
%%% End:
